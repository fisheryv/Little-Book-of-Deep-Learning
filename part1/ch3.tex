% !TeX root = ../main.tex
\chapter{训练}

如 \ref{sec1.1} 节所述,训练模型包括最小化损失 $\mathcal{L}(w)$,它反映了预测器 $f(\cdot;w)$ 在\keyterm{训练集} $\mathcal{D}$ 上的表现。

由于模型通常非常复杂,并且其表现与损失最小化程度直接相关,因此这里的最小化是一个关键挑战,涉及计算和数学难题。

\section{损失}

公式 \ref{eq1.1} 中的\keyterm{均方误差}示例是用于预测连续值的标准损失。

在密度建模中,标准损失是数据的似然度。如果 $f(x;w)$ 被解释为归一化的对数概率或对数密度,那么损失就是其值在训练样本上的总和的相反数,这对应于数据集的似然度。

\subsubsection*{交叉熵}

对于\keyterm{分类},通常的策略是模型的输出是一个向量,其中每个类别 $y$ 对应一个分量 $f(x;w)_y$,这被解释为非归一化概率的对数或 \keyterm{logit}。

如果 $X$ 为输入信号,$Y$ 为要预测的类别,我们可以根据 $f$ 计算\keyterm{后验概率}估计:
\[\hat{P}(Y=y \mid X=x) = \frac{\exp f(x;w)_y}{\sum_{z}\exp f(x;w)_z}\]
该表达式通常称为 logits 的 \keyterm{softmax},或更准确地说,称为 \keyterm{softargmax}。

为了与这种解释保持一致,模型应该被训练来最大化真实类别的概率,因此要最小化交叉熵,其表达式如下:
\begin{align*}
    \mathcal{L}_{ce}(w) &= -\frac{1}{N}\sum_{n=1}^{N} \log \hat{P}(Y=y_n \mid X=x_n) \\
    &= \frac{1}{N}\sum_{n=1}^{N} \underbrace{-\log \frac{\exp f(x_n;w)_{y_n}}{\sum_{z}\exp f(x_n;w)_z}}_{L_{ce}(f(x_n;w),y_n)}
\end{align*}

\subsubsection*{对比损失}

在某些设置中,即使要预测的值是连续的,监督也会采取排名约束的形式。这种情况的典型领域是\keyterm{度量学习},其目标是学习样本之间距离的度量,使得来自某个语义类别的样本 $x_a$ 与同一类别中的任意样本 $x_b$ 之间的距离都比来自另一个类别的任意样本 $x_c$ 之间的距离更近。例如,$x_a$ 和 $x_b$ 可以是某个人的两张照片,而 $x_c$ 则是另一个人的照片。

这种情况的标准方法是最小化\keyterm{对比损失},在这种情况下,例如,三元组 $(x_a,x_b,x_c)$,满足 $y_a = y_b \ne y_c$,求和
\[\max(0,1-f(x_a,x_c;w)+f(x_a,x_b;w))\]
除非 $f(x_a,x_c;w) \ge 1+f(x_a,x_b;w)$,否则该量将严格为正。

\subsubsection*{工程化损失}

通常,在训练期间最小化的损失并不是最终想要优化的实际量,而是一个代理量,是为了让找到最佳模型参数更为容易。例如,尽管实际的性能度量是分类错误率,但交叉熵是分类的标准损失,因为后者没有提供信息梯度,这是我们将在 \ref{sec3.3} 节中看到的关键要求。

还可以在损失中添加取决于模型本身的可训练参数项,以支持某些配置。

例如,\keyterm{权重衰减}正则化包括向损失中增加一个与参数平方和成比例的项。这可以被解释为在参数上施加了一个高斯贝叶斯先验,它偏好较小的值,从而减少了数据的影响。这会降低其在训练集上的表现,但会减少训练表现与新的、未见过的数据上的表现之间的差距。

\section{自回归模型}

自回归模型是一类关键方法,特别适用于处理自然语言处理和计算机视觉中的离散序列。

\subsubsection*{概率的链式法则}

这些模型使用概率论中的\keyterm{链式法则}:
\begin{align*}
    P(&X_1 = x_1,X_2 = x_2,\dots,X_T = x_T) = \\
    &P(X_1 = x_1) \\
    \times &P(X_2 = x_2 \mid X_1 = x_1) \\
    &\dots \\
    \times &P(X_T = x_T \mid X_1 = x_1,\dots,X_{T-1} = x_{T-1})
\end{align*}
尽管这种分解对于任何类型的随机序列都有效,但当感兴趣的信号是来自有限\keyterm{词汇表} $\{1, \dots ,K\}$ 的 \keyterm{Token} 序列时,它特别有效。

按照约定,附加 Token $\emptyset$ 代表``未知''量,我们可以将事件 ${X_1 = x_1,\dots,X_t = x_t}$ 表示为向量 $(x_1,\dots,x_t,\emptyset,\dots,\emptyset)$。则模型
\[f : \{\emptyset,1,\dots,K\}^T \to \mathbb{R}^K\]
在给定这样的输入的情况下计算与
\[\hat{P}(X_t \mid X_1 = x_1,\dots,X_{t-1} = x_{t-1})\]
相对应的 $K$ 个 \keyterm{logits} 的向量 $l_t$,允许在给定先前 Token 的情况下对一个 Token 进行采样。

链式法则确保在给定先前采样的 $x_1,\dots,x_{t-1}$ 的情况下,通过一次次地对第 $T$ 个 Token $x_t$ 进行采样,我们能够得到一个遵循联合分布的序列。这是一个\keyterm{自回归}生成模型。

训练这样的模型可以通过最小化训练序列和时间帧上的\keyterm{交叉熵损失}
\[Lce\big(f(x_1,\dots,x_{t-1},\emptyset,\dots,\emptyset;w),x_t\big)\]
之和来完成,这在形式上等同于最大化真实 $x_t$ 的似然。

传统上监测的值不是交叉熵本身,而是\keyterm[困惑度]{困惑度(Perplexity)},其定义为交叉熵的指数。它对应于具有相同熵的均匀分布的值的数量,这通常更易于解释。

\subsubsection*{因果模型}

我们所描述的训练过程对于每个 $t$ 都需要不同的输入,而且在 $t < t'$ 的情况下所做的大部分计算会在 $t'$ 时重复进行。这是极其低效的,因为 $T$ 通常是几百或几千的数量级。

解决这个问题的标准策略是设计一个模型 $f$ 一次性预测所有 logits 向量 $l_1,\dots,l_T$,即:
\[f : {1,\dots,K}^T \to \mathbb{R}^{T \times K}\]
但存在计算结构使得计算 $x_t$ 的 logits $l_t$ 仅依赖于输入值 $x_1,\dots,x_{t-1}$。

\begin{figure}
    \centering
    \includegraphics[width=0.9\textwidth]{fig/fig3.1.png}
    \caption[因果自回归模型]{如果输入序列的一个时间帧 $x_t$ 调节预测的 logits $l_s$ 只在 $s > t$ 时才有效,如蓝色箭头所示,则自回归模型 $f$ 是因果模型。这允许在训练期间一次性计算所有时间帧的分布。然而,在采样过程中,$l_t$ 和 $x_t$ 是顺序计算的,后者是用前者采样的,如红色箭头所示。}
    \label{fig3.1}
\end{figure}

这样的模型称为\keyterm{因果模型},因为在时间序列的情况下,它对应于不让未来影响过去,如图 \ref{fig3.1} 所示。

其结果是,每个位置上的输出都假设输入只在该位置之前可用的情况下所得到的。在训练过程中,这使得我们能够计算一个完整序列的输出,并最大化该序列所有 token 的预测概率,这又归结为最小化每个 token 的交叉熵之和。

请注意,为了简单起见,我们将 $f$ 定义为对长度固定为 $T$ 的序列进行操作。然而,实际使用的模型,例如我们将在 \ref{sec5.3} 节中看到的 Transformer 模型,能够处理任意长度的序列。

\subsubsection*{分词器(Tokenizer)}

处理自然语言时,一个重要技术细节是,token 的表示方法多种多样,从最细粒度的单个符号到整个单词,不一而足。而 token 表示的转换是由一个称为\keyterm{分词器}(\keyterm{tokenizer})的独立算法来完成的。

一种标准的方法是\keyterm{字节对编码}(\keyterm{Byte Pair Encoding},\keyterm{BPE})\cite{srivastava14a},它通过分层合并字符组来构造 token,尝试获取代表不同长度但频率相似的单词片段的 token,并将 token 分配给长的高频片段以及罕见的单个符号。

\section{梯度下降}\label{sec3.3}

\subsubsection*{学习率}

\subsubsection*{随机梯度下降}

\section{反向传播}

\subsubsection*{正向和反向传递}

\subsubsection*{资源利用}

\subsubsection*{梯度消失}

\section{深度值}\label{sec3.6}

\section{训练协议}\label{sec3.7}

\section{规模的好处}