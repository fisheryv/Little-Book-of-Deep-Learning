% -*- mode: latex; mode: reftex; mode: auto-fill; mode: flyspell; coding: utf-8; tex-command: "pdflatex.sh" -*-

\documentclass[oneside,11pt]{memoir}

% -*- mode: latex; mode: reftex; mode: auto-fill; mode: flyspell; coding: utf-8; -*-

%%%%%%%%%%%%%%%%%%%%%%%%%%%%%%%%%%%%%%%%%%%%%%%%%%%%%%%%%%%%%%%%%%%%%%%%%%%%%%%%%%%%%%%%%%%
\usepackage{ctex}
\usepackage{amsmath}
\usepackage{amssymb}
\usepackage{dsfont}
\usepackage{ifthen}
\usepackage{caption}

\let\ordinal\relax
\usepackage[us]{datetime}
\newdateformat{dotdate}{\THEYEAR.\twodigit{\THEMONTH}.\twodigit{\THEDAY}}

\usepackage{imakeidx}
\makeindex[columns=1]

\usepackage{enumitem}
\setlist[itemize]{leftmargin=0pt,itemindent=1em,itemsep=2ex}
\setlist{nosep} % or \setlist{noitemsep} to leave space around whole list

\usepackage[utf8]{inputenc}
\usepackage[T1]{fontenc}
\usepackage[osf]{libertine}
\usepackage{microtype}

\usepackage[
  linktocpage=true,
  unicode=true,
  bookmarks=true,
  bookmarksnumbered=false,
  bookmarksopen=false,
  breaklinks=true,
  pdfborder={0 0 1},
  backref=page,
  colorlinks=true,
  linkcolor=links,
  urlcolor=links,
  citecolor=links,
  hypertexnames=false, % to avoid errors with autonum
]{hyperref} % PDF meta-information specification

\urlstyle{same}

\usepackage[object=vectorian]{pgfornament}
\def\textsep{%
\vskip1.5ex

\centerline{\pgfornament[anchor=center,ydelta=0pt,width=2cm]{82}}

\vskip0.5ex
}

\AddToHook{cmd/section/before}{\clearpage}
\usepackage[section]{placeins}

\usepackage{xspace}
\def\wordfig{Figure\xspace}
\def\wordeq{Equation\xspace}
\def\wordtable{Table\xspace}
\def\wordchap{Chapter\xspace}

\let\oldcenter\center
\let\oldendcenter\endcenter
\renewenvironment{center}{\setlength\topsep{0pt}\oldcenter}{\oldendcenter}

\usepackage{environ}
\NewEnviron{hardcenter}{\makebox[\textwidth][c]{\BODY}}

%%%%%%%%%%%%%%%%%%%%%%%%%%%%%%%%%%%%%%%%%%%%%%%%%%%%%%%%%%%%%%%%%%%%%%%%%%%%%%%%%%%%%%%%%%%
% Math
%%%%%%%%%%%%%%%%%%%%%%%%%%%%%%%%%%%%%%%%%%%%%%%%%%%%%%%%%%%%%%%%%%%%%%%%%%%%%%%%%%%%%%%%%%%

\usepackage{amsmath}
\usepackage{amssymb}
\usepackage{dsfont}
\usepackage{mleftright}

\setlength{\thinmuskip}{1.5mu} % by default it is equal to 3 mu
\setlength{\medmuskip}{2mu} % by default it is equal to 4 mu
\setlength{\thickmuskip}{3.5mu} % by default it is equal to 5 mu

\makeatletter
\DeclareFontEncoding{LS1}{}{}
\DeclareFontSubstitution{LS1}{stix}{m}{n}
\DeclareMathAlphabet{\mathcal}{LS1}{stixscr}{m}{n}
\makeatother

%%%%%%%%%%%%%%%%%%%%%%%%%%%%%%%%%%%%%%%%%%%%%%%%%%%%%%%%%%%%%%%%%%%%%%

\newcommand{\inputgenerated}[1]{%
  \IfFileExists{#1}{\input{#1}}{%
    \errmessage{Cannot find "#1", compile with -shell-escape}\stop}%
}

%\newcommand{\gradient}[2]{{\nabla\!}_{#1 \mid #2}}
\newcommand{\gradient}[2]{{\nabla\!#1}_{\mid #2}}

\def\given{\,\middle\vert\,}
\newcommand{\proba}{{P}}
\newcommand{\seq}{{S}}
\newcommand{\expect}{\mathds{E}}
\newcommand{\variance}{\mathds{V}}
\newcommand{\empexpect}{\hat{\mathds{E}}}
\newcommand{\mutinf}{\mathds{I}}
\newcommand{\empmutinf}{\hat{\mathds{I}}}
\newcommand{\entropy}{\mathds{H}}
\newcommand{\empentropy}{\hat{\mathds{H}}}
\newcommand{\ganG}{\mathbf{G}}
\newcommand{\ganD}{\mathbf{D}}
\newcommand{\ganF}{\mathbf{F}}

\newcommand{\dkl}{\mathds{D}_{\mathsf{KL}}}
\newcommand{\djs}{\mathds{D}_{\mathsf{JS}}}

\newcommand*{\vertbar}{\rule[-1ex]{0.5pt}{2.5ex}}
\newcommand*{\horzbar}{\rule[.5ex]{2.5ex}{0.5pt}}

\def\positionalencoding{\operatorname{pos-enc}}
\def\concat{\operatorname{concat}}
\def\crossentropy{\LL_{\operatorname{ce}}}

\def\embedding{\operatorname{embed}}
\def\mha{\operatorname{mha}}
\def\layernorm{\operatorname{layernorm}}
\def\batchnorm{\operatorname{batchnorm}}
\def\fullyconnected{\operatorname{fully-conn}}
\def\softargmax{\operatorname{softargmax}}
\def\selfattention{\operatorname{self-att}}
\def\crossattention{\operatorname{cross-att}}
\def\attention{\operatorname{att}}
\def\relu{\operatorname{relu}}
\def\gelu{\operatorname{gelu}}
\def\dropout{\operatorname{dropout}}
\def\resblock{\operatorname{resblock}}
\def\dresblock{\operatorname{dresblock}}
\def\reshape{\operatorname{reshape}}
\def\convtwod{\operatorname{conv-2d}}
\def\maxpool{\operatorname{maxpool}}
\def\avgpool{\operatorname{avgpool}}
%\def\samax{\Upsilon}
%\def\samax{\operatorname{samax}}
\def\sigmoid{\operatorname{sigm}}
\def\sample{\operatorname{sample}}
\def\diag{\operatorname{diag}}
\def\sign{\operatorname{sign}}
\def\argmax{\operatornamewithlimits{argmax}}
\def\argmin{\operatornamewithlimits{argmin}}

%\usepackage{oldgerm}
\usepackage{relsize}

\usepackage{xfp}
\newcommand{\adaptedscale}[1]{#1}

%\newcommand{\li}[1]{^{\textgoth{#1}}}
\newcommand{\li}[1]{^{\scalebox{.5}{\textbf{#1}}}}
%% \newcommand{\li}[1]{^{\textbf{#1}}}
%\newcommand{\li}[1]{{|#1}}
\newcommand{\DATAVAR}{\mathbf{{\cal D}}}
\newcommand{\DATAVAL}{\mathbf{d}}
\newcommand{\BD}{\mathbf{D}}
\newcommand{\LL}{\mathcal{L}}
\newcommand{\Ll}{\mathcal{l}}
\newcommand{\RR}{\mathbb{R}}
\newcommand{\Lh}{\mathcal{h}}
\newcommand{\transpose}{^{\top}}

%%%%%%%%%%%%%%%%%%%%%%%%%%%%%%%%%%%%%%%%%%%%%%%%%%%%%%%%%%%%%%%%%%%%%%%%%%%%%%%%%%%%%%%%%%%
% tikz
%%%%%%%%%%%%%%%%%%%%%%%%%%%%%%%%%%%%%%%%%%%%%%%%%%%%%%%%%%%%%%%%%%%%%%%%%%%%%%%%%%%%%%%%%%%

\usepackage{tikz}
\usetikzlibrary{arrows,arrows.meta,calc}
\usetikzlibrary{patterns,backgrounds}
\usetikzlibrary{positioning,fit}
\usetikzlibrary{shapes.geometric,shapes.multipart}
\usetikzlibrary{patterns.meta,decorations.pathreplacing,calligraphy}
\usetikzlibrary{tikzmark}
\usetikzlibrary{decorations.pathmorphing}

% remove the "There is no ... in font nullfont!" errors
\AtBeginEnvironment{tikzpicture}{\tracinglostchars=0\relax}

%% \tikzset{
%% }

\definecolor{operatorcolor}{rgb}{0.95,0.95,1.00}
\definecolor{paramcolor}{rgb}{0.8,0.8,1.0}

\tikzset{
  axes/.style={
    samples=1000,
    %smooth,
    %scale=0.8,
  },
}

\newlength{\layergap}
\setlength{\layergap}{2pt}
\newlength{\layerthickness}
\setlength{\layerthickness}{12pt}
\newlength{\layerwidth}
\setlength{\layerwidth}{4.5em}

\newlength{\diminfoshift}
\setlength{\diminfoshift}{70pt}

\tikzset{
  >={Straight Barb[angle'=80,scale=\adaptedscale{1.2}]},
  deepnet/.style={
%%     background rectangle/.style={fill=paper},
%%     show background rectangle,
    %every text node part/.style={align=center},
    %rounded corners=0.5pt,
    curly brace/.style={sharp corners,very thick,decoration={calligraphic brace,amplitude=0.20cm},decorate},
    font=\footnotesize,
    halo/.style={
      %%       on layer=background,
      preaction={
        draw=white,line width=2pt,-,%shorten <=1pt,shorten >=1pt,
      },
    },
    operator/.style={draw=black!30,fill=operatorcolor,inner sep=1pt},
    next/.style={above=##1\layergap of \tikzlastnode},
    next/.default={1},
    prev/.style={below=##1\layergap of \tikzlastnode},
    prev/.default={1},
    var/.style={inner sep=2pt},
    flow/.style={thick},
    layer/.style={operator,minimum width=\layerwidth,minimum height=\layerthickness,text depth=1pt,text height=1.3ex},
    layer small/.style={layer,minimum width=\layerthickness},
    layer large/.style={layer,minimum height=1.5\layerthickness},
    layer very large/.style={layer,minimum height=1.75\layerthickness},
    info line/.style={
      draw=black,line width=0.4pt,dash pattern=on 0.4pt off 2pt,
%%       draw=black!50,line width=0.2pt,-,
      shorten >=2pt,shorten <=2pt,
    },
    block definition/.style={draw=black,inner sep=2\layergap,dash pattern=on 2.5pt off 0.5pt},
    replicated/.style={
      draw=black,
      inner sep=\layergap, dash pattern=on 2.5pt off 0.8pt,
      label={[%
          inner sep=2pt,
          anchor=south west,
        ]south east:$\times ##1$},
    },
    %
    inputs/.style={
      text depth=1.5ex,
      path picture={%
        \draw[black]
        ($(path picture bounding box.south west)+(1pt,6pt)$)--($(path picture bounding box.south east)+(-1pt,6pt)$)
        %
        node[midway,yshift=-15.5pt] {\scalebox{.5}{##1}};
      }
    },
    %
    param/.style={%
%      draw=paramcolor,
      fill=paramcolor,
%%       preaction={fill=white},
%%       pattern color=black!15,
%%       pattern={Lines[line width=0.5pt,angle=-45,distance=1pt]}
    },
    meta param/.style={label={[%
          inner sep=0pt,
          text depth=0pt,
          anchor=south west,
          shift={(1.5pt,0pt)},
        ]south east:{\tiny\color{blue}##1}}},
  }
}

\newcommand{\diminfo}[3]{%
  \coordinate (t) at ($(#2.north)+(\diminfoshift,0.5\layergap)$);
  \node[inner sep=0pt,yshift=-0.5pt] (s) at (#1.north east-|t) {\tiny #3};
  \draw[info line] (#1.north east|-s)--(s);
}

\newcommand{\defop}[2]{%
%%   \coordinate (BL) at ($(#1.north)+(-0.49\textwidth, 4\layergap)$);
%%   \coordinate (BR) at ($(#1.north)+( 0.49\textwidth, 4\layergap)$);
%%   \coordinate (TL) at ($(#2.south-|#1)+(-0.49\textwidth,-4\layergap)$);
%%   \coordinate (TR) at ($(#2.south-|#1)+( 0.49\textwidth,-4\layergap)$);
  \begin{pgfinterruptboundingbox}
    \node[anchor=south west,inner sep=2pt] (label) at #1 {#2};
    \draw[decorate,decoration={coil,amplitude=0.5pt,segment length=2pt,aspect=0}] (label.south west) -- (label.south east);
  \end{pgfinterruptboundingbox}
}

%%%%%%%%%%%%%%%%%%%%%%%%%%%%%%%%%%%%%%%%
% style on layer

\tikzset{%
  on layer/.code={
    \pgfonlayer{#1}\begingroup
    \aftergroup\endpgfonlayer
    \aftergroup\endgroup
  }}

\makeatletter
%% fix for bb computation of double wires.
%% from https://tex.stackexchange.com/questions/130456/tikz-double-lines-are-shifted
\tikzset{
  only coordinates are relevant/.is choice,
  only coordinates are relevant/.default=true,
  only coordinates are relevant/true/.code={%
    \tikz@addmode{\pgf@relevantforpicturesizefalse}},
  only coordinates are relevant/false/.code={%
    \tikz@addmode{\pgf@relevantforpicturesizetrue}}
}
\makeatother

%%%%%%%%%%%%%%%%%%%%%%%%%%%%%%%%%%%%%%%%

\makeatletter
% extract interval `start:end` values
\def\get@interval@start#1:#2\@nil{#1}
\def\get@interval@end#1:#2\@nil{#2}
% get domain
\def\domainmin{\expandafter\get@interval@start\tikz@plot@domain\@nil}
\def\domainmax{\expandafter\get@interval@end\tikz@plot@domain\@nil}
% get range
\def\rangemin{\expandafter\get@interval@start\tikz@plot@range\@nil}
\def\rangemax{\expandafter\get@interval@end\tikz@plot@range\@nil}
\makeatother

\usepackage{pgfplots}
\usepgfplotslibrary{patchplots,colormaps}
\pgfplotsset{compat = newest}

\newcommand{\mygrid}[5]{%
  \pgfmathsetmacro{\xmin}{#1+1}
  \pgfmathsetmacro{\xmax}{#1+#3-1}
  \pgfmathsetmacro{\ymin}{#2+1}
  \pgfmathsetmacro{\ymax}{#2+#4-1}
  \ifthenelse{\equal{#5}{}}
  {\draw (#1,#2) rectangle ++(#3,#4);}
  {\draw[fill=#5] (#1,#2) rectangle ++(#3,#4);}
  \foreach \x in {\xmin,...,\xmax}{
    \draw (\x,#2)-- ++(0,#4);
  }
  \foreach \y in {\ymin,...,\ymax}{
    \draw (#1,\y)-- ++(#3,0);
  }
}

\newcommand{\amatrix}[7]{%
  \begin{tikzpicture}[scale=\adaptedscale{0.2}]
    \ifthenelse{\equal{#7}{}}
               {}
               {\draw[draw=none,fill=#7] (#3,#4) rectangle ++(#5,#6);}
               \mygrid{0}{0}{#1}{#2}{}
  \end{tikzpicture}%
}

\newcommand{\gridcube}[3]{% 7,4,6

  \foreach \b in { 0,...,#2 }{
    \draw (0,\b,0)--++(#1,0,0)--++(0,0,#3);
  }

  \foreach \d in { 0,...,#1 }{
    \draw (\d,0,0)--++(0,#2,0)--++(0,0,#3);
  }

  \foreach \hw in { 0,...,#3 }{
    \draw (0,0,\hw)++(#1,0,0)--++(0,#2,0)--++(-#1,0,0);
  }
}

%%%%%%%%%%%%%%%%%%%%%%%%%%%%%%%%%%%%%%%%%%%%%%%%%%%%%%%%%%%%%%%%%%%%%%%%%%%%%%%%%%%%%%%%%%%
% Bibliography
%%%%%%%%%%%%%%%%%%%%%%%%%%%%%%%%%%%%%%%%%%%%%%%%%%%%%%%%%%%%%%%%%%%%%%%%%%%%%%%%%%%%%%%%%%%

\usepackage[square]{natbib}
\bibliographystyle{plainnatmodified}
\nobibintoc
\newcommand{\biburl}[1]{\href{#1}{pdf}}

%%%%%%%%%%%%%%%%%%%%%%%%%%%%%%%%%%%%%%%%%%%%%%%%%%%%%%%%%%%%%%%%%%%%%%%%%%%%%%%%%%%%%%%%%%%
% Layout
%%%%%%%%%%%%%%%%%%%%%%%%%%%%%%%%%%%%%%%%%%%%%%%%%%%%%%%%%%%%%%%%%%%%%%%%%%%%%%%%%%%%%%%%%%%

\setlength{\cftbeforepartskip}{3ex}
\setlength{\cftbeforechapterskip}{1.0ex}
\setlength{\cftbeforesectionskip}{0.1ex}

%% \setsecnumdepth{subsection}
%% \renewcommand{\thesubsection}{\alph{subsection}\,-\hskip -12pt\,}
%% \setsecnumformat{\csname the#1\endcsname :}

\cftsetindents{part}{0em}{1.8em}
\cftsetindents{chapter}{0em}{1.8em}
\cftsetindents{section}{1.8em}{2.2em}

\setlength{\parindent}{0cm}
\setlength{\parskip}{2ex}

\setstocksize{15cm}{8cm}
\settrimmedsize{\stockheight}{\stockwidth}{*}
\setlrmarginsandblock{8pt}{8pt}{*}
\setulmarginsandblock{14pt}{26pt}{*}
\setheadfoot{14pt}{14pt}
\setheaderspaces{*}{*}{*}
%% \setlength{\headsep}{0pt}
%% \setlength{\headheight}{0pt}

%% \newcommand\ignoreme[1]{}
%% \setsecheadstyle{\ignoreme}

%% \makepagestyle{littlebook}
%% \makeoddhead{littlebook}{}{}{}
%% \makeevenhead{littlebook}{}{}{}
\newcommand{\myfooter}{\footnotesize {\thepage \hskip 0.8em \raisebox{-2pt}{\vline height 8pt} \hskip 0.4em \thelastpage}}
%% \makeoddfoot{littlebook}{}{\myfooter}{}
%% \makeevenfoot{littlebook}{}{\myfooter}{}
\makeoddfoot{plain}{}{\myfooter}{}
\makeevenfoot{plain}{}{\myfooter}{}
\pagestyle{plain}

%%%%%%%%%%%%%%%%%%%%%%%%%%%%%%%%%%%%%%%%%%%%%%%%%%%%%%%%%%%%%%%%%%%%%%%%%%%%%%%%%%%%%%%%%%%

\renewcommand{\partnamefont}{\centering\sffamily\scshape\Huge}
\renewcommand{\partnumfont}{\sffamily\Huge}
\renewcommand{\parttitlefont}{\centering\sffamily\scshape\Huge}
\renewcommand{\beforepartskip}{\vspace*{\stretch{3}}}
\renewcommand{\afterpartskip}{%
\vspace*{\stretch{4}}
\newpage%
}

%%%%%%%%%%%%%%%%%%%%%%%%%%%%%%%%%%%%%%%%%%%%%%%%%%%%%%%%%%%%%%%%%%%%%%%%%%%%%%%%%%%%%%%%%%%

\makechapterstyle{Tufte}{
\renewcommand{\chapterheadstart}{\null \vskip1.5\onelineskip}
\renewcommand{\printchaptername}{\large\sffamily\itshape\chaptername}
\renewcommand{\printchapternum}{\LARGE\thechapter \\}
\renewcommand{\afterchapternum}{}
\renewcommand{\printchaptertitle}[1]{
\raggedright
\itshape\Huge{##1}}
\renewcommand{\afterchaptertitle}{
\vskip3\onelineskip
}}
\chapterstyle{Tufte}

%%%%%%%%%%%%%%%%%%%%%%%%%%%%%%%%%%%%%%%%%%%%%%%%%%%%%%%%%%%%%%%%%%%%%%%%%%%%%%%%%%%%%%%%%%%

\setsecheadstyle{\sethangfrom{\noindent ##1}\raggedright\sffamily\itshape\Large}
\setbeforesecskip{-.9\onelineskip}
\setaftersecskip{.75\onelineskip}

\setsubsecheadstyle{\sethangfrom{\noindent  ##1}\raggedright\sffamily\itshape\large}
\setbeforesubsecskip{\onelineskip}
\setaftersubsecskip{.65\onelineskip}

\setsubsubsecheadstyle{\sethangfrom{\noindent ##1}\raggedright\sffamily\itshape}
\setbeforesubsubsecskip{-.5\onelineskip}
\setaftersubsubsecskip{.1\onelineskip}

%%%%%%%%%%%%%%%%%%%%%%%%%%%%%%%%%%%%%%%%%%%%%%%%%%%%%%%%%%%%%%%%%%%%%%%%%%%%%%%%%%%%%%%%%%%

\captiontitlefont{\itshape\small}
\captionnamefont{\small}
\newcommand{\likecaption}{\color{black}\itshape\small}

\midsloppy

\checkandfixthelayout

%%%%%%%%%%%%%%%%%%%%%%%%%%%%%%%%%%%%%%%%%%%%%%%%%%%%%%%%%%%%%%%%%%%%%%%%%%%%%%%%%%%%%%%%%%%
% The \todo command
\newcounter{nbdrafts}
\setcounter{nbdrafts}{0}
\makeatletter
\newcommand{\checknbdrafts}{
\ifnum \thenbdrafts > 0
\@latex@warning@no@line{*WARNING* The document contains \thenbdrafts \space draft note(s)}
\fi}
\newcommand{\todo}[1]{\addtocounter{nbdrafts}{1}{\color{red} #1}}
\makeatother
%%%%%%%%%%%%%%%%%%%%%%%%%%%%%%%%%%%%%%%%%%%%%%%%%%%%%%%%%%%%%%%%%%%%%%%%%%%%%%%%%%%%%%%%%%%
\definecolor{paper}{rgb}{0.95,0.95,0.95}
\definecolor{math}{rgb}{0.0,0.5,0.0}
%\definecolor{links}{rgb}{0.0,0.2,0.5}
\definecolor{links}{rgb}{0.0,0.2,0.85}
%\definecolor{hlcolor}{rgb}{0.8,1.0,0.85}

\definecolor{blue}{rgb}{0.3,0.5,0.85}
\definecolor{red}{rgb}{0.65,0.0,0.0}
\definecolor{green}{rgb}{0.0,0.50,0.0}
\definecolor{dimmed}{rgb}{0.8,0.8,0.8}
\definecolor{orange}{rgb}{1.0,0.75,0.0}

%%%%%%%%%%%%%%%%%%%%%%%%%%%%%%%%%%%%%%%%%%%%%%%%%%%%%%%%%%%%%%%%%%%%%%
% Pretty underline, taken from
% https://tex.stackexchange.com/questions/36894/underline-omitting-the-descenders

\usepackage{soul}
\usepackage{xcolor}
\usepackage{xparse}
\makeatletter

\ExplSyntaxOn
\cs_new:Npn \white_text:n #1
  {
    \fp_set:Nn \l_tmpa_fp {#1 * .01}
    \llap{\textcolor{white}{\the\SOUL@syllable}\hspace{\fp_to_decimal:N \l_tmpa_fp em}}
    \llap{\textcolor{white}{\the\SOUL@syllable}\hspace{-\fp_to_decimal:N \l_tmpa_fp em}}
  }
\NewDocumentCommand{\whiten}{ m }
    {
      \int_step_function:nnnN {1}{1}{#1} \white_text:n
    }
\ExplSyntaxOff

\NewDocumentCommand{ \prettyul }{ D<>{5} O{0.2ex} O{0.1ex} +m } {%
\begingroup
\setul{#2}{#3}%
\def\SOUL@uleverysyllable{%
   \setbox0=\hbox{\the\SOUL@syllable}%
   \ifdim\dp0>\z@
      \SOUL@ulunderline{\phantom{\the\SOUL@syllable}}%
      \whiten{#1}%
      \llap{%
        \the\SOUL@syllable
        \SOUL@setkern\SOUL@charkern
      }%
   \else
       \SOUL@ulunderline{%
         \the\SOUL@syllable
         \SOUL@setkern\SOUL@charkern
       }%
   \fi}%
    \ul{#4}%
\endgroup
}

\makeatother

% end of prettyul
%%%%%%%%%%%%%%%%%%%%%%%%%%%%%%%%%%%%%%%%%%%%%%%%%%%%%%%%%%%%%%%%%%%%%%

\usepackage{accsupp}
\usepackage{xcolor, soul}

\definecolor{hlcolor}{rgb}{1.0,1.0,0.5}
\sethlcolor{hlcolor}
%% \definecolor{ulcolor}{rgb}{0.65,0.65,0.65}
%% \setulcolor{ulcolor}

%% \index{Attention Layer@\hypertarget{Attention Layer.ind}{}Attention Layer}
%% \href{\#Attention Layer.ind}%

%% \newcommand{\keyterm}[2][]{%
%%   \ifthenelse{\equal{#1}{}}
%%              {\prettyul[2pt]{#2}\linkedindex{#2}}
%%              {\prettyul[2pt]{#2}\linkedindex{#1}}%
%% }

\newcommand{\keytermold}[2][]{%
%  \BeginAccSupp{method=plain,ActualText={#2}}%
  \ifthenelse{\equal{#1}{}}
             {\prettyul[2pt]{#2}\index{#2}}
             {\prettyul[2pt]{#2}\index{#1}}%
%  \EndAccSupp{}%
}

\setulcolor{black}
\setul{0.3ex}{0.5pt}
\newcommand{\keyterm}[2][]{%
  \ul{#2}%
%%   \prettyul[2pt]{#2}%
  \ifthenelse{\equal{#1}{}}{\index{#2}}{\index{#1}}%
}

%%%%%%%%%%%%%%%%%%%%%%%%%%%%%%%%%%%%%%%%%%%%%%%%%%%%%%%%%%%%%%%%%%%%%%

\newcommand{\gizmo}{%
  \begin{tikzpicture}[scale=0.25]
    \draw[draw=none,fill=red]  (0,0) rectangle ++(1,1);
    \draw[draw=none,fill=blue] (1,0) rectangle ++(1,1);
    \draw[draw=none,fill=blue] (0,1) rectangle ++(1,1);
    \draw[draw=none,fill=red]  (1,1) rectangle ++(1,1);
  \end{tikzpicture}
}

%%%%%%%%%%%%%%%%%%%%%%%%%%%%%%%%%%%%%%%%%%%%%%%%%%%%%%%%%%%%%%%%%%%%%%%%%%%%%%%%%%%%%%%%%%%


\usepackage{lipsum} 

\hypersetup{
  pdfauthor={Fisher Yu},
  pdftitle={Little Book of Deep Learning Chinese Version},
  pdfsubject={},
  pdfkeywords={Deep Learning, Mathematics, Chinese},
  pdfproducer={LaTeX and TikZ},
  pdfcreator={XeLaTex},
}

%%%%%%%%%%%%%%%%%%%%%%%%%%%%%%%%%%%%%%%%%%%%%%%%%%%%%%%%%%%%%%%%%%%%%%%%%%%%%%%%%%%%%%%%%%%

\begin{document}

\thispagestyle{empty}

\begin{center}

\vspace*{\stretch{1}}

{\huge 深度学习\\[0.75ex] 随身宝书}

\vspace*{4ex}

作者:François Fleuret \\
译者:Fisher Yu

\vspace*{\stretch{1}}

\begin{figure}[!htbp]
\centering
\includegraphics[width=0.8\textwidth]{cover.png}
\end{figure}

\vspace*{\stretch{1}}

\footnotesize \dotdate\today

\end{center}

\newpage

%%%%%%%%%%%%%%%%%%%%%%%%%%%%%%%%%%%%%%%%%%%%%%%%%%%%%%%%%%%%%%%%%%%%%%%%%%%%%%%%%%%%%%%%%%%

\vspace*{\stretch{1.25}}

\href{https://fleuret.org/francois/}{François Fleuret} 是瑞士日内瓦大学计算机科学教授。

封面插图是新认知机(Neocognitron) 的示意图,来自 Fukushima [1980],它是深度神经网络的重要前身。

本书格式适配手机屏幕。

\vspace*{\stretch{1}}

\vspace*{-3ex}

\newpage

%%%%%%%%%%%%%%%%%%%%%%%%%%%%%%%%%%%%%%%%%%%%%%%%%%%%%%%%%%%%%%%%%%%%%%%%%%%%%%%%%%%%%%%%%%%
% Table of content
%%%%%%%%%%%%%%%%%%%%%%%%%%%%%%%%%%%%%%%%%%%%%%%%%%%%%%%%%%%%%%%%%%%%%%%%%%%%%%%%%%%%%%%%%%%

{
\everymath{\color{black}}
\tableofcontents % Prints the table of contents
%\addcontentsline{toc}{chapter}{Contents}
}

\clearpage

\listoffigures*
\addcontentsline{toc}{chapter}{插图列表}

%%%%%%%%%%%%%%%%%%%%%%%%%%%%%%%%%%%%%%%%%%%%%%%%%%%%%%%%%%%%%%%%%%%%%%%%%%%%%%%%%%%%%%%%%%%

\chapter*{前言}
\addcontentsline{toc}{chapter}{前言}

\citep{nips-1502.c399862d3b9d6b76c8436e924a68c45b} 证明,仅用二十多年前[\cite[LeCun et al., 1989]{lecun-89e}]诞生的结构简单的\emph{人工神经网络},只需放大百倍,并在同等放大的数据集上进行训练,就可以以巨大优势击败当时最先进的复杂图像识别方法。这一发现引发了当今人工智能的进步。

这一突破得益于\emph{图形处理单元} (\emph{GPU})、大众市场以及原本为实时图像合成而开发现在重新应用于人工神经网络的高度并行计算设备。

从那时起,在``\emph{深度学习}''这一总称下,各种网络结构、训练策略和专用硬件的创新使得人工神经网络的规模和训练所用数据量呈指数级增长 [\ref{}{Sevilla et al., 2022}]。这掀起了从计算机视觉与机器人科学到语音和自然语言处理等一系列技术领域的成功应用浪潮。

尽管深度学习的大部分内容并不难理解,但它综合了线性代数、微积分、概率、优化、信号处理、编程、算法和高性能计算等不同的学科知识,使得学习变得复杂。

这本小书不会试图详尽无遗,而是仅限于理解重要模型所需的背景知识。事实证明,这是一种广受好评的方法,本书在 Twitter 上发布一个月内,PDF 版本的下载量就达到了 25 万次。

如果您没有从官网 
\begin{center}
\href{https://fleuret.org/public/lbdl.pdf}{https://fleuret.org/public/lbdl.pdf}
\end{center}
获取本书,请从官网下载,以便我可以统计读者的数量。

\begin{flushright}
  François Fleuret,\\
  2023.06.23
\end{flushright}

%%%%%%%%%%%%%%%%%%%%%%%%%%%%%%%%%%%%%%%%%%%%%%%%%%%%%%%%%%%%%%%%%%%%%%%%%%%%%%%%%%%%%%%%%%%

\part{基础}

% !TeX root = ../main.tex
\chapter{机器学习}

从历史上看,\underline{深度学习}属于更广泛的统计\underline{机器学习}领域,因为从本质上它关注从数据中学习表示的方法。其所涉及的技术最初来自\underline{人工神经网络},``深度''一词强调模型是长映射组合,目前已经验证可以获得更好的表现。

深度模型的模块化、多功能性和可扩展性催生了大量特定的数学方法和软件开发工具,使深度学习成为一个独特而广阔的技术领域。

\section{从数据中学习}

从数据训练模型最简单的用例是当信号 $x$ 输入时,例如车牌图片,人们想要从中预测量 $y$,例如车牌上的字符串。

在许多现实世界场景中,$x$ 是在不受控制的环境中捕获的高维信号,因此给出将 $x$ 和 $y$ 相关联的分析方法十分复杂。

我们能做的就是收集一个由 $(x_n, y_n)$ 对组成的大型\underline{训练集} $\mathcal{D}$,并设计一个\underline{参数模型} $f$。它是一段计算机代码,其中包含可调节其行为的\underline{可训练参数} $w$,因此,使用适当的 $w^*$ 值,就会得到一个良好的预测器。这里的``良好''意味着如果将 $x$ 输入这段代码,则它计算的值 $\hat{y}= f(x;w^*)$ 是对 $y$ 的良好估计,如果有的话,该估计与训练集中的 $x$ 相关联。

这里``良好''的概念通常用\underline{损失} $\mathcal{L}(w)$ 来形式化,当 $f(\cdot;w)$ 在 $\mathcal{D}$ 上表现良好时,损失会很小。然后,\underline{训练}模型就是计算使 $\mathcal{L}(w^*)$ 最小的 $w^*$ 值。本书的大部分内容都是关于 $f$ 的定义,在现实场景中,$f$ 是预先定义子模块的复杂组合。

组成 $w$ 的可训练参数通常称为\underline{权重},类似于生物神经网络的突触权重。除了这些参数之外,模型通常还依赖于\underline{元参数},这些元参数是根据领域先验知识、最佳实践或资源限制设置的。它们也可以以某种方式进行优化,但使用的技术不同于用于优化 $w$ 的技术。

\section{基函数回归}\label{sec1.2}

我们通过一个简单的例子来说明模型的训练,其中 $x_n$ 和 $y_n$ 为两个实数,损失为\underline{均方误差}:
\begin{equation}
    \mathcal{L}(w) = \frac{1}{N}\sum_{n=1}^{N}\big(y_n-f(x_n;w)\big)^2
\end{equation}
$f(\cdot;w)$ 是预定义基函数 $f_1, \dots , f_K$ 的线性组合,其中 $w =(w_1, \dots ,w_K)$:
\[f(x;w) = \sum_{k=1}^{K}w_kf_k(x)\]
\begin{figure}
    \centering
    \includegraphics[width=0.9\textwidth]{fig/fig1.1.png}
    \caption{给定基函数基础(蓝色曲线)和训练集(黑点),我们可以计算前者的最佳线性组合(红色曲线),以近似后者的均方误差。}
    \label{fig1.1}
\end{figure}
由于 $f(x_n;w)$ 相对于 $w_k$ 是线性的,并且 $\mathcal{L}(w)$ 相对于 $f(x_n;w)$ 是二次的,因此损失 $\mathcal{L}(w)$ 相对于 $w_k$ 是二次的,$w^*$ 最小化损失归结为求解线性系统。请参阅图 \ref{fig1.1} 中以高斯核为 $f_k$ 的示例。

\section{欠拟合和过拟合}

一个关键要素是模型\underline{能力}与训练数据之间的相互作用,即模型的灵活性和适应不同数据的能力与训练数据的数量和质量之间的相互作用。当能力不足时,模型无法拟合数据,导致训练时误差较高。这被称为\underline{欠拟合}。

相反,当数据量不足时,如图 \ref{fig1.2} 所示,模型通常会学到特定于训练样本的特征,从而在训练过程中获得优异的表现,
\begin{figure}
    \centering
    \includegraphics[width=0.9\textwidth]{fig/fig1.2.png}
    \caption{如果训练数据量(黑点)与模型能力相比较小,则训练期间拟合模型的经验表现(红色曲线)反映出对底层数据结构(细黑色曲线)的实际拟合很差,因此其预测的有效性不佳。}
    \label{fig1.2}
\end{figure}
但代价是与数据的全局结构拟合较差,并且在新输入上表现不佳。这种现象称为\underline{过拟合}。

因此,应用\underline{机器学习}艺术很大一部分在于设计不太灵活但仍然能够适应数据的模型。这是通过在模型中设计正确的\underline{归纳偏置}来实现的,这意味着其结构对应于手头数据的底层结构。

尽管这种经典观点与合理大小的深度模型相关,而对于具有大量可训练参数和极端能力但在预测方面仍表现良好的大型模型来说,事情会变得混乱。我们将在 \ref{sec3.6} 节和 \ref{sec3.7} 节再次讨论这一点。

\section{模型分类}

我们可以将\underline{机器学习}模型的使用分为三类:

\begin{itemize}
    \item \underline{回归}包括预测一个连续值向量 $y \in \mathbb{R}^K$,例如,给定一个输入信号 $X$,预测一个物体的几何位置。这是我们在 \ref{sec1.2} 节所学内容的多维泛化。训练集由一对输入信号和\underline{基本事实}值组成。
    \item \underline{分类}旨在是从有限集合 ${1,\dots,C}$ 中预测值,例如,预测图像 $X$ 的标签 $Y$。与回归一样,训练集由一对输入信号和基本事实量组成,基本事实量在这里是该集合中的一个标签。解决这个问题的标准方法是为每个潜在类别预测一个分数,使得正确的类别具有最高分数。
    \item \underline{密度建模}的目标是对数据 $\mu X$ 本身(例如图像)的概率密度函数进行建模。在这种情况下,训练集由值 $x_n$ 组成,没有要预测的关联量,并且训练后的模型能够评估概率密度函数,或从分布中采样,或两者兼而有之。
\end{itemize}

回归和分类通常被称为\underline{监督学习},因为必须提供在训练过程中需要预测的值,例如由人类专家提供。相反,密度建模通常被视为\underline{无监督学习},因为它只需要使用现有数据而无需生成相关的基本事实。

这三个类别并不是互不相干的;例如,分类可以被转换为类别分数回归,或者离散序列密度建模可以被看作迭代分类。此外,它们并不涵盖所有情况。有时我们可能希望预测复合量,或者多个类别,或者在信号条件下对密度进行建模。
% !TeX root = ../main.tex
\chapter{高性能计算}

从实现的角度来看,深度学习涉及使用大量数据执行繁重的计算。\keyterm{图形处理单元} (\keyterm{GPU}) 可以在经济实惠的硬件上运行此类计算,从而对该领域的成功发挥了至关重要的作用。

使用 GPU 的重要性,以及由此产生的对有效计算的技术限制,迫使该领域的研究不断平衡数学的合理性和新方法的可实施性。

\section{GPU、TPU 和批处理}\label{sec2.1}

图形处理单元最初是为实时图像合成而设计的,这需要高度并行的架构,而这种架构恰好非常适合深度模型。随着人工智能用途的增加,GPU 配备了专用的\keyterm{张量核心},并且开发了深度学习专用芯片,例如谷歌的\keyterm{张量处理单元}(\keyterm{TPU})。

GPU 拥有数千个并行单元和自己的快速内存。限制因素通常不是计算单元的数量,而是\keyterm{对内存的读写操作}。最慢的链接位于 CPU 内存和 GPU 内存之间,因此应避免跨设备复制数据。此外,GPU 本身的结构涉及多级\keyterm{缓存},这些缓存容量更小但速度更快,并且应该组织计算以避免不同缓存之间的复制。

具体来说,就是通过将计算组织为完全适合 GPU 内存的\keyterm{样本批次}再通过并行处理来实现的。当操作器组合样本和模型参数时,两者都必须移动到实际计算单元附近的高速缓存中。分批进行仅允许复制模型参数一次,而不是为每个样本复制一次。实际上,GPU 处理适合内存的批次几乎与处理单个样本一样快。

标准 GPU 的理论\keyterm{峰值性能}为每秒 $10^{13}-10^{14}$ 次浮点运算 (\keyterm{FLOP}),其内存通常在 $8$ 到 $80$ GB 之间。标准 \keyterm{FP32} 采用 $32$ 位编码浮点数,但经验结果表明,使用 $16$ 位编码,甚至对某些操作数采用更低的编码,不会降低性能。

我们将在 \ref{sec3.7} 节中讨论深度架构的尺寸。

\section{张量}

GPU 与 PyTorch 或 JAX 等\keyterm{深度学习框架}通过将要处理的量组织为\keyterm{张量}来操作要处理的量,张量是沿多个离散轴排列的一系列标量。它们是 $\mathbb{R}^{N_1 \times \dots \times N_D}$ 的元素,泛化了向量和矩阵的概念。

张量用于表示要处理的信号、模型的\keyterm{可训练参数}以及它们计算的中间量。后者被称为\keyterm{激活},指的是神经元激活。

例如,时间序列自然地被编码为 $T \times D$ 张量,或者,由于历史原因,编码为 $D \times T$ 张量,其中 $T$ 是其持续时间,$D$ 是每个时间帧特征表示的维度,通常称为\keyterm{通道}数。类似地,二维结构信号可以表示为 $D \times H \times W$ 张量,其中 $H$ 和 $W$ 是其高度和宽度。RGB 图像对应于 $D = 3$,但在大型模型中通道数可能会多至到数千个。

添加更多维度可以表示一系列对象。例如,$50$ 个分辨率为 $32 \times 24$ 的 RGB 图像可以编码为 $50 \times 3 \times 24 \times 32$ 张量。

深度学习库提供了大量操作,包括标准线性代数、复杂重塑和提取以及深度学习特定操作,其中一些我们将在第 \ref{ch4} 章中看到。张量的实现将形状表示与内存中系数的存储布局分开。这允许在不复制系数的情况下完成许多重塑、转置和提取操作,因此速度非常快。

在实践中,几乎任何计算都可以分解为基本张量运算,这避免了语言级别的非并行循环和不良的内存管理。

除了方便使用之外,张量还有助于提高计算效率。所有参与开发可操作深度模型的人员,从驱动程序、库和模型的设计者到计算机和芯片的设计者,都知道数据将作为张量进行操作。由此产生的对局部性和块可分解性的限制使该链条中的所有参与者都能给出最佳设计。
% !TeX root = ../main.tex
\chapter{训练}

如 \ref{sec1.1} 节所述,训练模型包括最小化损失 $\mathcal{L}(w)$,它反映了预测器 $f(\cdot;w)$ 在\keyterm{训练集} $\mathcal{D}$ 上的表现。

由于模型通常非常复杂,并且其表现与损失最小化程度直接相关,因此这里的最小化是一个关键挑战,涉及计算和数学难题。

\section{损失}

公式 \ref{eq1.1} 中的\keyterm{均方误差}示例是用于预测连续值的标准损失。

在密度建模中,标准损失是数据的似然度。如果 $f(x;w)$ 被解释为归一化的对数概率或对数密度,那么损失就是其值在训练样本上的总和的相反数,这对应于数据集的似然度。

\subsubsection*{交叉熵}

对于\keyterm{分类},通常的策略是模型的输出是一个向量,其中每个类别 $y$ 对应一个分量 $f(x;w)_y$,这被解释为非归一化概率的对数或 \keyterm{logit}。

如果 $X$ 为输入信号,$Y$ 为要预测的类别,我们可以根据 $f$ 计算\keyterm{后验概率}估计:
\[\hat{P}(Y=y \mid X=x) = \frac{\exp f(x;w)_y}{\sum_{z}\exp f(x;w)_z}\]
该表达式通常称为 logits 的 \keyterm{softmax},或更准确地说,称为 \keyterm{softargmax}。

为了与这种解释保持一致,模型应该被训练来最大化真实类别的概率,因此要最小化交叉熵,其表达式如下:
\begin{align*}
    \mathcal{L}_{ce}(w) &= -\frac{1}{N}\sum_{n=1}^{N} \log \hat{P}(Y=y_n \mid X=x_n) \\
    &= \frac{1}{N}\sum_{n=1}^{N} \underbrace{-\log \frac{\exp f(x_n;w)_{y_n}}{\sum_{z}\exp f(x_n;w)_z}}_{L_{ce}(f(x_n;w),y_n)}
\end{align*}

\subsubsection*{对比损失}

在某些设置中,即使要预测的值是连续的,监督也会采取排名约束的形式。这种情况的典型领域是\keyterm{度量学习},其目标是学习样本之间距离的度量,使得来自某个语义类别的样本 $x_a$ 与同一类别中的任意样本 $x_b$ 之间的距离都比来自另一个类别的任意样本 $x_c$ 之间的距离更近。例如,$x_a$ 和 $x_b$ 可以是某个人的两张照片,而 $x_c$ 则是另一个人的照片。

这种情况的标准方法是最小化\keyterm{对比损失},在这种情况下,例如,三元组 $(x_a,x_b,x_c)$,满足 $y_a = y_b \ne y_c$,求和
\[\max(0,1-f(x_a,x_c;w)+f(x_a,x_b;w))\]
除非 $f(x_a,x_c;w) \ge 1+f(x_a,x_b;w)$,否则该量将严格为正。

\subsubsection*{工程化损失}

通常,在训练期间最小化的损失并不是最终想要优化的实际量,而是一个代理量,是为了让找到最佳模型参数更为容易。例如,尽管实际的性能度量是分类错误率,但交叉熵是分类的标准损失,因为后者没有提供信息梯度,这是我们将在 \ref{sec3.3} 节中看到的关键要求。

还可以在损失中添加取决于模型本身的可训练参数项,以支持某些配置。

例如,\keyterm{权重衰减}正则化包括向损失中增加一个与参数平方和成比例的项。这可以被解释为在参数上施加了一个高斯贝叶斯先验,它偏好较小的值,从而减少了数据的影响。这会降低其在训练集上的表现,但会减少训练表现与新的、未见过的数据上的表现之间的差距。

\section{自回归模型}

自回归模型是一类关键方法,特别适用于处理自然语言处理和计算机视觉中的离散序列。

\subsubsection*{概率的链式法则}

这些模型使用概率论中的\keyterm{链式法则}:
\begin{align*}
    P(&X_1 = x_1,X_2 = x_2,\dots,X_T = x_T) = \\
    &P(X_1 = x_1) \\
    \times &P(X_2 = x_2 \mid X_1 = x_1) \\
    &\dots \\
    \times &P(X_T = x_T \mid X_1 = x_1,\dots,X_{T-1} = x_{T-1})
\end{align*}
尽管这种分解对于任何类型的随机序列都有效,但当感兴趣的信号是来自有限\keyterm{词汇表} $\{1, \dots ,K\}$ 的 \keyterm{Token} 序列时,它特别有效。

按照约定,附加 Token $\emptyset$ 代表``未知''量,我们可以将事件 ${X_1 = x_1,\dots,X_t = x_t}$ 表示为向量 $(x_1,\dots,x_t,\emptyset,\dots,\emptyset)$。则模型
\[f : \{\emptyset,1,\dots,K\}^T \to \mathbb{R}^K\]
在给定这样的输入的情况下计算与
\[\hat{P}(X_t \mid X_1 = x_1,\dots,X_{t-1} = x_{t-1})\]
相对应的 $K$ 个 \keyterm{logits} 的向量 $l_t$,允许在给定先前 Token 的情况下对一个 Token 进行采样。

链式法则确保在给定先前采样的 $x_1,\dots,x_{t-1}$ 的情况下,通过一次次地对第 $T$ 个 Token $x_t$ 进行采样,我们能够得到一个遵循联合分布的序列。这是一个\keyterm{自回归}生成模型。

训练这样的模型可以通过最小化训练序列和时间帧上的\keyterm{交叉熵损失}
\[Lce\big(f(x_1,\dots,x_{t-1},\emptyset,\dots,\emptyset;w),x_t\big)\]
之和来完成,这在形式上等同于最大化真实 $x_t$ 的似然。

传统上监测的值不是交叉熵本身,而是\keyterm[困惑度]{困惑度(Perplexity)},其定义为交叉熵的指数。它对应于具有相同熵的均匀分布的值的数量,这通常更易于解释。

\subsubsection*{因果模型}

我们所描述的训练过程对于每个 $t$ 都需要不同的输入,而且在 $t < t'$ 的情况下所做的大部分计算会在 $t'$ 时重复进行。这是极其低效的,因为 $T$ 通常是几百或几千的数量级。

解决这个问题的标准策略是设计一个模型 $f$ 一次性预测所有 logits 向量 $l_1,\dots,l_T$,即:
\[f : {1,\dots,K}^T \to \mathbb{R}^{T \times K}\]
但存在计算结构使得计算 $x_t$ 的 logits $l_t$ 仅依赖于输入值 $x_1,\dots,x_{t-1}$。

\begin{figure}
    \centering
    \includegraphics[width=0.9\textwidth]{fig/fig3.1.png}
    \caption[因果自回归模型]{如果输入序列的一个时间帧 $x_t$ 调节预测的 logits $l_s$ 只在 $s > t$ 时才有效,如蓝色箭头所示,则自回归模型 $f$ 是因果模型。这允许在训练期间一次性计算所有时间帧的分布。然而,在采样过程中,$l_t$ 和 $x_t$ 是顺序计算的,后者是用前者采样的,如红色箭头所示。}
    \label{fig3.1}
\end{figure}

这样的模型称为\keyterm{因果模型},因为在时间序列的情况下,它对应于不让未来影响过去,如图 \ref{fig3.1} 所示。

其结果是,每个位置上的输出都假设输入只在该位置之前可用的情况下所得到的。在训练过程中,这使得我们能够计算一个完整序列的输出,并最大化该序列所有 token 的预测概率,这又归结为最小化每个 token 的交叉熵之和。

请注意,为了简单起见,我们将 $f$ 定义为对长度固定为 $T$ 的序列进行操作。然而,实际使用的模型,例如我们将在 \ref{sec5.3} 节中看到的 Transformer 模型,能够处理任意长度的序列。

\subsubsection*{分词器(Tokenizer)}

处理自然语言时,一个重要技术细节是,token 的表示方法多种多样,从最细粒度的单个符号到整个单词,不一而足。而 token 表示的转换是由一个称为\keyterm{分词器}(\keyterm{tokenizer})的独立算法来完成的。

一种标准的方法是\keyterm{字节对编码}(\keyterm{Byte Pair Encoding},\keyterm{BPE})\cite{srivastava14a},它通过分层合并字符组来构造 token,尝试获取代表不同长度但频率相似的单词片段的 token,并将 token 分配给长的高频片段以及罕见的单个符号。

\section{梯度下降}\label{sec3.3}

\subsubsection*{学习率}

\subsubsection*{随机梯度下降}

\section{反向传播}

\subsubsection*{正向和反向传递}

\subsubsection*{资源利用}

\subsubsection*{梯度消失}

\section{深度值}\label{sec3.6}

\section{训练协议}\label{sec3.7}

\section{规模的好处}

%%%%%%%%%%%%%%%%%%%%%%%%%%%%%%%%%%%%%%%%%%%%%%%%%%%%%%%%%%%%%%%%%%%%%%%%%%%%%%%%%%%%%%%%%%%

\part{深度模型}

% !TeX root = ../main.tex
\chapter{模型构成}\label{ch4}

深度模型只不过是复杂的张量计算,最终可以用线性代数和数学分析分解为标准数学运算。多年来,该领域开发了大量语义清晰的高级模块以及由这些模块组合而来的复杂模型,这些模型已被证明在特定应用领域非常有效。

经验证据和理论结果表明,更深的架构(即长映射组合)可以获得更好的表现。正如我们在 \ref{sec3.4} 节中看到的,由于\keyterm{梯度消失},训练这样的模型具有挑战性,而多项重要技术贡献缓解了这个问题。

\section{层的概念}\label{sec4.1}

我们将那些被设计出来并通过经验认定为通用且高效的标准复杂复合张量操作称为\keyterm{层}。这些层通常包含可训练参数,并且对于设计和描述大型深度模型来说,它们提供了一个便捷的粒度级别。这个术语来源于简单多层神经网络,尽管现代模型可能采用此类模块的复杂图形形式,并包括多个并行路径。

\begin{figure}[h]
    \centering
    \includegraphics[width=0.9\textwidth]{fig/fig4.0.png}
\end{figure}

在接下来的几页中,我将遵循上面所示模型绘制的约定:

\begin{itemize}
    \item 算子/层被绘制为框,
    \item 深色表示嵌入了可训练的参数,
    \item 没有默认值的元参数用蓝色字添加在右侧,
    \item 带有乘法因子的虚线外框表示一组层按顺序复制,每个层都有自己的一组可训练参数(如果有的话),
    \item 在某些情况下,当输出维度与输入维度不同时,会在右侧标明。
\end{itemize}

此外,具有复杂内部结构的层会用高度更高的框来表示。

\section{线性层}\label{sec4.2}

就计算和参数数量而言,最重要的模块是\keyterm{线性层}。它们得益于数十年来在矩阵运算算法和芯片设计方面的研究与工程进步。

请注意,在深度学习中,``线性''这一术语通常不恰当地指代\keyterm{仿射运算},即一个线性表达式和一个常数偏置之和。

\subsubsection*{全连接层}

最基本的线性层是\keyterm{全连接层},由大小为 $D' \times D$ 的可训练权重矩阵 $W$ 和维度为 $D'$ 的偏置向量 $b$ 参数化。它实现了泛化到任意张量形状的仿射变换,其中补充的维度被解释为向量索引。形式上,给定维度为 $D_1 \times \dots \times D_K \times D$ 的输入 $X$,它计算出维度为 $D_1 \times \dots \times D_K \times D'$ 的输出 $Y$,其中
\begin{align*}
    \forall d_1,\dots,d_K&,\\
    Y[d_1&,\dots,d_K] = WX[d_1,\dots,d_K]+b
\end{align*}
虽然乍看之下,这种仿射运算似乎仅限于旋转、对称和平移等几何变换,但实际上它能做的远不止这些。特别是,用于降维或信号过滤的投影,而且,从点积作为相似性度量的角度来看,矩阵-向量乘积可以解释为计算输入向量所编码的查询与矩阵行所编码的键之间的匹配得分。

正如我们在 \ref{sec3.3} 节中看到的,梯度下降从\keyterm{参数的随机初始化}开始。如果这一操作做得过于简单,如 \ref{sec3.4} 节所示,网络可能会遭受激活和梯度爆炸或消失的影响 \citep{glorot10a}。深度学习框架实现了初始化方法,特别是按照输入的维度来缩放随机参数,以保持激活的方差恒定并防止病态行为。

\subsubsection*{卷积层}

线性层可以将任意形状的张量通过重塑成向量的方式作为输入,只要它具有正确数量的系数即可。然而,这样的层不太适合处理大型张量,因为参数数量和操作数量与输入和输出维度的乘积成正比。例如,要处理一个大小为 $256 \times 256$ 的 RGB 图像作为输入并计算相同大小的结果,将需要大约 $4 \times 10^{10}$ 个参数和乘法运算。

除了这些实际问题之外,大多数高维信号都是强结构化的。例如,图像在平移、缩放和某些对称性方面表现出短期相关性和统计平稳性。这并没有反映在全连接层的\keyterm{归纳偏置}中,它完全忽略了信号结构。

为了利用这些规律,首选的工具是\keyterm{卷积层}。卷积层同样是仿射的,但它局部处理时间序列或二维信号,并在各处使用相同的操作符。

\keyterm{一维卷积}主要由三个\keyterm{元参数}定义:内核大小 $K$、输入通道数 $D$、输出通道数 $D'$,以及仿射映射 $\phi(\cdot;w):\mathbb{R}^{D \times K} \to \mathbb{R}^{D' \times 1}$ 的可训练参数 $w$。

它可以处理任何大小为 $D \times T$ 且 $T \ge K$ 的张量 $X$,并将 $\phi(\cdot;w)$ 应用于 $X$ 的每个大小为 $D \times K$ 的子张量,将结果存储在大小为 $D' \times (T-K+1)$ 的张量 $Y$ 中,如图 \ref{fig4.1}(左半部分)所示。

\newpage

\begin{figure}[h]
    \centering
    \includegraphics[width=0.9\textwidth]{fig/fig4.1.png}
    \caption[一维卷积]{一维卷积(左)接受 $D \times T$ 张量 $X$ 作为输入,将相同的仿射映射 $\phi(\cdot;w)$ 应用于形状为 $D \times K$ 的每个子张量,并将生成的 $D' \times 1$ 张量存储到 $Y$ 中。一维转置卷积(右)接受 $D \times T$ 张量作为输入,将相同的仿射映射 $\phi(\cdot;w)$ 应用于每个形状为 $D \times 1$ 的子张量,并对偏移后的 $D' \times K$ 结果张量求和。两者都可以处理不同大小的输入。}
    \label{fig4.1}
\end{figure}

\keyterm{二维卷积}与之类似,但具有 $K \times L$ 大小的内核,并接受 $D \times H \times W$ 大小的张量作为输入(参见图 \ref{fig4.2},左半部分)。

\begin{figure}[h]
    \centering
    \includegraphics[width=0.9\textwidth]{fig/fig4.2.png}
    \caption[二维卷积]{二维卷积(左)接受 $D \times H \times W$ 张量 $X$ 作为输入,将相同的仿射映射 $\phi(\cdot;w)$ 应用于形状为 $D \times K \times L$ 的每个子张量,并将生成的 $D' \times 1 \times 1$ 张量存储到 $Y$ 中。二维 转置卷积(右)接受 $D \times H \times W$ 张量作为输入,将相同的仿射映射 $\phi(\cdot;w)$ 应用于每个形状为 $D \times 1 \times 1$ 的子张量,并对偏移后的 $D' \times K \times L$ 结果张量求和得到 $Y$。}
    \label{fig4.2}
\end{figure}



这两种操作的可训练参数都是 $\phi$ 的参数,可以分别将其设想为大小为 $D \times K$ 或 $D \times K \times L$ 的 $D'$ 个\keyterm{过滤器},以及一个维度为 $D'$ 的\keyterm{偏置向量}。

这样的层对平移是\keyterm{等变}的,这意味着如果输入信号被平移,输出也会以类似的方式变换。当处理其分布对于平移不变的信号时,此属性会产生理想的\keyterm{归纳偏差}。

卷积层还接受三个额外的\keyterm{元参数},如图 \ref{fig4.3} 所示:

\begin{itemize}
    \item \keyterm{填充}指定在处理输入张量之前应在输入张量周围添加多少个零系数,特别是在内核大小大于 $1$ 时维持张量尺寸。其默认值为 $0$。
    \item \keyterm{步幅}指定在处理输入时使用的步长,允许通过使用大步长以几何方式减小输出大小。其默认值为 $1$。
    \item \keyterm{膨胀}指定局部仿射操作符的过滤器系数之间的索引计数。其默认值为 $1$,更大的值对应于在系数之间插入零,这会增加过滤器/内核的大小,同时保持可训练参数数量不变。
\end{itemize}

\begin{figure}
    \centering
    \includegraphics[width=0.9\textwidth]{fig/fig4.3.png}
    \caption[步长、填充和膨胀]{除了内核大小和输入/输出通道数之外,卷积还接受三个元参数:步长 $s$(左)在经过输入张量时调节步长,填充 $p$(右上)指定在处理输入张量之前在输入张量周围添加多少个零元素,膨胀 $d$(右下)参数化过滤器系数之间的索引计数。}
    \label{fig4.3}
\end{figure}

除了通道数之外,卷积的输出通常小于其输入。在没有填充或膨胀的一维情况下,如果输入的大小为 $T$,内核的大小为 $K$,步幅为 $S$,则输出的大小为 $T' = (T - K)/S + 1$。

\newpage

给定由卷积层计算的激活,或某个位置上所有通道的值向量,它所依赖的输入信号部分称为其\keyterm{感受野}(见图 \ref{fig4.4})。与 $D \times H \times W$ 激活张量的单个通道对应的 $H \times W$ 子张量之一称为\keyterm{激活图}。

\begin{figure}
    \centering
    \includegraphics[width=0.9\textwidth]{fig/fig4.4.png}
    \caption[感受野]{给定一系列卷积层(这里为红色)中的激活,其\keyterm{感受野}是输入信号(蓝色)中调节其值的区域。每个中间卷积层大约按照核的宽度和高度增加该区域的宽度和高度。}
    \label{fig4.4}
\end{figure}

卷积用于重新组合信息,通常是为了减少表示的空间大小,以换取更多数量的通道,从而转化为更丰富的局部表示。它们可以实现微分算子,例如边缘检测器或模板匹配机制。一系列这样的层也可以视为一种组合和分层表示 \citep{arxiv-1311.2901},或者作为一个扩散过程,其中信息在穿过层时可以通过内核大小的一半进行传输。

逆运算是\keyterm{转置卷积},也由局部仿射算子组成,由与卷积类似的元和可训练参数定义,例如,在一维情况下,它将一个仿射映射 $\phi(\cdot;w):R^{D \times 1} \to R^{D' \times K}$ 应用于输入的每个 $D \times 1$ 子张量,并将偏移后的 $D' \times K$ 结果张量求和以计算其输出。这样的操作符增加了信号的尺寸,直观上可以理解为一个合成过程(见图 \ref{fig4.1} 和图 \ref{fig4.2} 的右半部分)。

一系列卷积层是将大维度信号(如图像或声音样本)映射到低维张量的常用架构。例如,这可以用来获取用于分类的类别分数或压缩表示。转置卷积层以相反的方式用来从压缩表示构建大维度信号,要么是为了评估压缩表示是否包含足够的信息来重构信号,要么是为了合成,因为在低维表示上学习密度模型更容易。我们将在 \ref{sec5.2} 节中重新讨论这个话题。

\section{激活函数}\label{sec4.3}

如果网络仅组合线性组件,那么它本身就只是线性算子,因此让网络具有非线性运算非常必要。这些\keyterm{非线性运算}主要是通过\keyterm{激活函数}来实现的,激活函数是将输入张量的每个分量单独通过一个映射进行转换的层,从而得到一个相同形状的张量。

有许多不同的激活函数,但最常用的是\keyterm{线性整流函数}(\keyterm{ReLU}) \citep{glorot11a},它将负值设置为零并保持正值不变(见图 \ref{fig4.5},右上)
$$
\text{relu}(x) = \begin{cases}
    0 &\text{如果}\; x < 0 \\
    x &\text{如果}\; x \ge 0
 \end{cases}
 $$
 鉴于深度学习的核心训练策略依赖于梯度,因此一个在零点不可微且在数轴正半轴为常数的映射似乎是有问题的。然而,梯度下降所需的主要属性是梯度平均具有信息性。训练开始时,参数初始化和数据归一化使一半的激活为正,从而确保了这一点。

 \begin{figure}
    \centering
    \includegraphics[width=0.9\textwidth]{fig/fig4.5.png}
    \caption[激活函数]{激活函数。}
    \label{fig4.5}
\end{figure}

在 ReLU 被普遍使用之前,标准激活函数是\keyterm{双曲正切函数}(\keyterm{Tanh},见图 \ref{fig4.5},左上),它在负半轴和正半轴都会以指数速度快速饱和,这加剧了梯度消失。

其他流行的激活函数遵循相同的思路,即保持正值不变压缩负值。\keyterm{Leaky ReLU} \citep{relu_hybrid_icml2013_final} 对负值应用一个小的正乘法因子(见图 \ref{fig4.5},左下):
$$
\text{leakyrelu}(x) = \begin{cases}
    ax &\text{如果}\; x < 0 \\
    \enspace x &\text{如果}\; x \ge 0
 \end{cases}
 $$
 而 \keyterm{GELU} \citep{arxiv-1606.08415} 是利用高斯分布的累积分布函数来定义的,即:
 \[\text{gelu}(x) = xP(Z \le x)\]
 其中 $Z \sim \mathcal{N} (0,1)$。它的行为大致类似于平滑的 ReLU(见图 \ref{fig4.5},右下)。

 激活函数的选择,特别是 ReLU 变体的选择,通常是由经验表现驱动的。

\section{池化}\label{sec4.4}

减少信号大小的一种经典策略是使用\keyterm{池化}操作,将多个激活合并为一个理想情况下能总结信息的激活。此类操作中最标准的是\keyterm{最大池化}层,它与卷积类似,可以在一维和二维中操作,并由\keyterm{内核大小}定义。

在其标准形式中,该层在空间大小等于内核大小的非重叠子张量上计算每个通道的最大激活。这些值存储在与输入具有相同通道数的结果张量中,并且其空间大小能被内核大小整除。与卷积一样,该算子具有三个\keyterm{元参数}:\keyterm{填充}、\keyterm{步长}和\keyterm{膨胀},默认情况下步长等于内核大小。遵循与卷积相同的公式(参见第 \ref{sec4.2} 节),较小的步长会产生较大的结果张量。

最大操作可以直观地解释为逻辑析取,或者,当它经过一系列通过计算局部分数来表示部件存在的\keyterm{卷积层}时,作为一种编码方式编码至少有一个部件实例存在。它牺牲了精确位置,从而使其不受局部变形的影响。

一个常见的替代方案是\keyterm{平均池化}层,它计算子张量上的平均值而不是最大值。这是一种线性操作,而最大池化则不是。

\begin{figure}
    \centering
    \includegraphics[width=0.9\textwidth]{fig/fig4.6.png}
    \caption[最大池化]{一维最大池化将 $D \times T$ 大小的张量 $X$ 作为输入,计算非重叠 $1 \times L$ 子张量(蓝色)的最大值,并将结果值(红色)存储在 $D \times (T / L)$ 大小的张量 $Y$ 中。}
    \label{fig4.6}
\end{figure}

\section{Dropout}\label{sec4.5}

某些层被专门设计用来促进训练或改进学学习得到的表示。

这类主要贡献之一是 \keyterm{Dropout} \citep{srivastava14a}。这种层没有可训练参数,但有一个元参数 $p$,并接受任意形状的张量作为输入。

通常在测试期间会关闭 Dropout,这种情况下其输出等于输入。当它处于激活状态时,它有概率 $p$ 独立地将输入张量的每个激活置为零,并且以 $\frac{1}{1-p}$ 因子重新缩放所有激活以保持期望值不变(见图 \ref{fig4.7}) 。

\begin{figure}
    \centering
    \includegraphics[width=0.9\textwidth]{fig/fig4.7.png}
    \caption[Dropout]{Dropout 可以处理任意形状的张量。在训练期间(左),它以概率 $p$ 将激活随机设置为零,并应用乘法因子来保持预期值不变。在测试期间(右),它保持所有激活不变。}
    \label{fig4.7}
\end{figure}

使用 Dropout 的动机在于促进有意义的单个激活,并阻止群体表征。由于一组 $k$ 个激活通过 Dropout 层保持完整的概率为 $(1-p)^k$,因此联合表征变得不可靠,从而使训练过程避免使用它们。Dropout 也可以被视为一种噪声注入,使训练更加稳健。

在处理图像和二维张量时,信号的短期相关性和由此产生的冗余抵消了 Dropout 的影响,因为可以从其邻居中推断出设置为零的激活。 因此,二维张量的 Dropout 将整个通道设置为零,而不是单个激活(见图 \ref{fig4.8})。

\begin{figure}
    \centering
    \includegraphics[width=0.9\textwidth]{fig/fig4.8.png}
    \caption[二维 Dropout]{诸如图像之类的二维信号通常表现出很强的短期相关性,并且可以从其邻居中推断出各个激活。这种冗余消除了标准非结构化 Dropout 的影响,因此二维张量的常用 Dropout 层会丢弃整个通道而不是单个值。}
    \label{fig4.8}
\end{figure}

尽管 Dropout 通常用于改进训练而在推理过程中处于非活动状态,但它可以在某些设置中用作随机化策略,例如,根据经验估计置信度分数 \citep{arxiv-1506.02142}。

\section{归一化层}\label{sec4.6}

在深度架构训练中,一类重要的操作是\keyterm{归一化层},它强制对一组激活函数的经验平均值和方差进行标准化。

这一类别的主要层是\keyterm{批量归一化} \citep{icml43442},它是唯一一个处理整批数据而不是单个样本的标准层。它由元参数 $D$ 和两组可训练标量参数 $\beta_1,\dots,\beta_D$ 和 $\gamma_1,\dots,\gamma_D$ 参数化。

给定一个由 $B$ 个 $D$ 维样本 $x_1,\dots,x_B$ 组成的批数据,它首先计算每个 D 分量的经验平均值 $\hat{m}_d$ 和方差 $\hat{v}_d$:
\begin{align*}
    \hat{m}_d &= \frac{1}{B}\sum_{b=1}^{B}x_{b,d} \\
    \hat{v}_d &= \frac{1}{B}\sum_{b=1}^{B}(x_{b,d}-\hat{m}_d)^2
\end{align*}
从中计算每个分量 $x_{b,d}$ 的归一化值 $z_{b,d}$,经验平均值为 $0$,方差为 $1$,并由此得出最终结果值 $y_{b,d}$,平均值为 $\beta_d$,标准差为 $\gamma_d$:
\begin{align*}
    \forall b, \quad z_{b,d} &= \frac{x_{b,d}-\hat{m}_d}{\sqrt{\hat{v}_d+\epsilon}}\\
    y_{b,d} &= \gamma_dz_{b,d}+\beta_d
\end{align*}
由于此标准化是跨批次定义的,因此仅在训练期间完成。在测试过程中,该层根据整个训练集上偏移平均值估计的 $m_d$ 和 $v_d$ 来转换各个样本,这可以归结为每个组件的固定仿射变换。

批量归一化背后的动机是避免网络早期层在训练期间的缩放变化影响到后续所有层,这些层随后需要相应地调整它们的可训练参数。尽管实际的作用机制可能比这一初衷更加复杂,但这种层显著地简化了深度模型的训练过程。

在二维张量的情况下,为了遵循卷积层处理所有位置的相似性原则,归一化是按通道进行的,覆盖所有二维位置,而 $\beta$ 和 $\gamma$ 仍然是 $D$ 维向量,因此缩放/偏移不依赖于二维位置。因此,如果待处理张量的形状为 $B \times D \times H \times W$,对于 $d = 1,\dots,D$,该层根据相应的 $B \times H \times W$ 切片计算 $(\hat{m}_d,\hat{v}_d)$,对其进行归一化。最后使用可训练参数 $\beta_d$ 和 $\gamma_d$ 缩放和偏移其分量。

\begin{figure}
    \centering
    \includegraphics[width=0.9\textwidth]{fig/fig4.9.png}
    \caption[批量归一化]{批量归一化(左)对给定 $d$ 的每组激活的均值和方差进行归一化,并使用每个 $d$ 的学习参数缩放/偏移同一组激活。层归一化(右)对特定 $b$ 的每组激活进行归一化,并使用由相同索引的学习参数对给定 $d,h,w$ 的每组激活进行缩放/偏移。}
    \label{fig4.9}
\end{figure}

因此,给定一个 $B \times D$ 张量,批量归一化会在 $b$ 上对其进行归一化,并根据 $d$ 对其进行缩放/偏移,这可以通过 $\gamma$ 的逐元素乘积和与 $\beta$ 的求和来实现。给定 $B \times D \times H \times W$ 张量,它会根据 $b,h,w$ 进行归一化,并根据 $d$ 进行缩放/偏移(参见图 \ref{fig4.9},左图)。

这种方法可以根据这些维度进行泛化。例如,\keyterm{层归一化} \citep{arxiv-1607.06450} 计算单个样本所有分量的矩,并进行归一化,然后对各个分量分别进行缩放和偏移(参见图 \ref{fig4.9},右图)。因此,给定一个 $B \times D$ 的张量,它会根据 $d$ 进行归一化,并且根据相同的维度进行缩放/偏移。给定一个 $B \times D \times H \times W$ 的张量,它会根据 $d,h,w$ 进行归一化,并且根据同样的维度进行缩放/偏移。

\newpage

与批量归一化不同的是,由于层归一化单独处理每个样本,因此它在训练和测试期间的表现是相同的。

\section{跳跃连接}\label{sec4.7}

另一种缓解梯度消失问题并允许深层架构训练的技术是\keyterm{跳跃连接} \citep{arxiv-1411.4038, arxiv-1505.04597}。它本身并不是层,而是一种架构设计,在该设计中,某些层的输出会原样传输到模型中更深的其他层,绕过中间的处理。这个未经修改的信号可以与连接分支进入的层的输入进行连接或相加(参见图 \ref{fig4.10})。一种特殊类型的跳跃连接是\keyterm{残差连接},它通过求和来结合信号,并且通常只跳过几个层(参见图 \ref{fig4.10},右侧)。

这种设计最理想的特性是,确保即使在某个阶段出现了梯度消除的处理情况,梯度仍然能够通过跳过连接进行传播。特别是残差连接,它允许构建多达数百层的深度模型和关键模型,例如计算机视觉中的\keyterm{残差网络} \citep{arxiv-1512.03385}(详见 \ref{sec5.2} 节)和自然语言处理中的 \keyterm{Transformers} \citep{arxiv-1706.03762}(详见 \ref{sec5.3} 节),完全都是由带有残差连接的层块组成的。

\newpage

\begin{figure}
    \centering
    \includegraphics[width=0.9\textwidth]{fig/fig4.10.png}
    \caption[跳跃连接]{在这张图中,用红色突出显示的跳过连接将信号不变地跨越多个层传递。一些架构(中间)会缩小再放大表征的尺寸,以便在多个尺度上操作,它们采用跳过连接将网络早期部分的输出送至后续在相同尺度上操作的层 \citep{arxiv-1411.4038, arxiv-1505.04597}。而残差连接(右侧)是一种特殊的跳过连接,它将原始信号与转换后的信号相加,通常最多绕过几个层 \citep{arxiv-1512.03385}。}
    \label{fig4.10}
\end{figure}

跳跃连接还可以促进模型的多尺度推理,通过连接具有兼容大小的层,在重新放大信号大小之前减小信号大小,例如用于\keyterm{语义分割}(详见 \ref{sec6.4} 节)。在残差连接的情况下,它们还可以通过简化学习任务,使其变为寻找一个增量改进而不是完全更新,从而促进学习。

\section{注意力层}\label{sec4.8}

在许多应用中,需要一种操作能够结合张量中相隔较远位置的局部信息。例如,在\keyterm{图像合成}中为了生成连贯且真实的细节,或者在\keyterm{自然语言处理}中为了做出语法或语义决策,需要结合段落中不同位置的单词。

\keyterm{全连接层}无法处理高维信号,也无法处理可变大小的信号,并且\keyterm{卷积层}无法快速传播信息。例如,通过在大空间区域上求平均值来聚合卷积结果的策略会受到将多个信号混合到有限数量维度的困扰。

\keyterm{注意力层}通过计算结果张量的每个分量到输入张量的每个分量的注意力得分来专门解决这个问题,没有局部性约束,并相应地对整个张量的特征进行平均 \citep{arxiv-1706.03762}。

尽管注意力层比其他层复杂得多,但它已然成为许多最新模型的标准元素。特别是,它是 \keyterm{Transformer} 的关键构建块,Transformer 是\keyterm{大语言模型}的主导架构。参见 \ref{sec5.3} 和 \ref{sec7.1} 节。

\subsubsection*{注意力算子}

给定
\begin{itemize}
    \item $Q$ 为具有 $N^Q \times D^{QK}$ 大小的\keyterm{查询}张量
    \item $K$ 为具有 $N^{KV} \times D^{QK}$ 大小的\keyterm{键}张量
    \item $V$ 为具有 $N^{KV} \times D^V$ 大小的\keyterm{值}张量
\end{itemize}
\keyterm{注意力算子}计算 $N^Q \times D^V$ 维张量
\[Y = \text{att}(K,Q,V)\]
为此,它首先为每个查询索引 $q$ 和每个键索引 $k$ 计算一个注意力得分 $A_{q,k}$,作为查询 $Q_q$ 和键点积的 \keyterm{softargmax}:
\begin{equation}
    A_{q,k} = \frac{\exp\Big(\frac{1}{\sqrt{D^{QK}}}Q_q \cdot K_k\Big)}{\sum_{l}^{}\exp\Big(\frac{1}{\sqrt{D^{QK}}}Q_q \cdot K_l\Big)} \label{eq4.1}
\end{equation}
其中缩放因子 $\frac{1}{\sqrt{D^{QK}}}$ 即使对于较大的 $D^{QK}$ 也能保持值的范围大致不变。

然后,根据注意力得分对值进行加权平均,为每个查询计算出一个检索值(参见图 \ref{fig4.11}):
\begin{equation}
    Y_q = \sum_{K}^{}A_{q,k}V_k \label{eq4.2}
\end{equation}

\begin{figure}
    \centering
    \includegraphics[width=0.9\textwidth]{fig/fig4.11.png}
    \caption[注意力算子解析]{注意力算子可以解释为将每个查询 $Q_q$ 与所有键 $K_1, \dots ,K_{N^{KV}}$ 进行匹配,以获得归一化注意力得分 $A_{q,1}, \dots ,A_{q,N^{KV}}$(见左图和公式 \ref{eq4.1}),然后将值 $V_1, \dots ,V_{N^{KV}}$ 根据这些得分进行加权平均,计算出结果 $Y_q$(见右图和公式 \ref{eq4.2})。}
    \label{fig4.11}
\end{figure}

因此,如果查询 $Q_n$ 与某个键 $K_m$ 的匹配程度远远高于所有其他键,则相应的注意力得分 $A_{n,m}$ 将接近于 $1$,而检索到的值 $Y_n$ 将是与该键关联的值 $V_m$。但是,如果它与多个键匹配程度相当,则 $Y_n$ 将是这些相关键关联值的平均值。

这可以实现为
\[\text{att}(Q,K,V) = \underbrace{\text{softargmax}\bigg(\frac{QK^T}{\sqrt{D^{QK}}}\bigg)V}_{A}\]

\begin{figure}
    \centering
    \includegraphics[width=0.9\textwidth]{fig/fig4.12.png}
    \caption[完全注意力算子]{注意力算子 $Y = \text{att}(Q,K,V)$ 首先计算注意力矩阵 $A$ 作为 $QK^T$ 的每个查询的 softargmax,它可以在归一化之前被常量矩阵 $M$ 屏蔽。该注意力矩阵在乘以 $V$ 以获得结果 $Y$ 之前先经过一个 dropout 层。通过将 $M$ 对角线下方全置为 $1$,对角线上方全置为 $0$,可以使该算子具有\keyterm{因果性}。}
    \label{fig4.12}
\end{figure}

如图 \ref{fig4.12} 所示,该操作通常以两种方式扩展。首先,可以在 softargmax 归一化之前将注意力矩阵乘以布尔矩阵 $M$ 来屏蔽注意力矩阵。例如,这允许通过取 $M$ 对角线以下全为 $1$ 而对角线以上全为 $0$ 来使算子具有\keyterm{因果性},从而防止 $Y_q$ 依赖于索引 $k$ 大于 $q$ 的键和值。其次,注意力矩阵在乘以 $V$ 之前由 \keyterm{dropout 层}(参见 \ref{sec4.5} 节)进行处理,从而在训练期间提供通常的好处。

\subsubsection*{多头注意力层}

\begin{figure}
    \centering
    \includegraphics[width=0.9\textwidth]{fig/fig4.13.png}
    \caption[多头注意力层]{多头注意力层对其每一个头 $h = 1,\dots,H$,都对输入序列 $X^Q, X^K, X^V$ 的各个元素施加一个参数化的线性变换,得到将由注意力算子处理以计算 $Y_h$ 的序列 $Q,K,V$。这 $H$ 个序列沿特征连接在一起,个别元素通过最后一个线性算子,得到最终的结果序列 $Y$。}
    \label{fig4.13}
\end{figure}

这个无参数注意力算子是图 \ref{fig4.13} 所示的多头注意力层中的关键元素。该层的结构由几个元参数定义:头的数量 $H$ 以及三组 $H$ 个可训练权重矩阵
\begin{itemize}
    \item $W^Q$ 尺寸为 $H \times D \times D^{QK}$
    \item $W^K$ 尺寸为 $H \times D \times D^{QK}$
    \item $W^V$ 尺寸为 $H \times D \times D^V$
\end{itemize}
分别计算输入中的查询、键和值,以及最终的权重矩阵 $W^O$,其大小为 $HD^V \times D$,用于聚合每个头的结果。

它以三个序列
\begin{itemize}
    \item $X^Q$ 尺寸为 $N^Q \times D$
    \item $X^K$ 尺寸为 $N^{KV} \times D$
    \item $X^V$ 尺寸为 $N^{KV} \times D$
\end{itemize}
作为输入,并从中进行计算,对于 $h = 1,\dots,H$,
\[Y_h = \text{att}\Big(X^QW_h^Q, X^KW_h^K, X^VW_h^V\Big)\]
该序列 $Y_1,\dots,Y_H$ 沿特征维度连接,并将所得序列的每个单独元素乘以 $W^O$ 以获得最终结果:
\[Y = \Big(Y_1 \mid \dots \mid Y_H\Big)W^O\]
正如我们将在 \ref{sec5.3} 节和图 \ref{fig5.6} 中看到的,该层用于构建两个模型子结构:\keyterm{自注意力块}(其中三个输入序列 $X^Q$、$X^K$ 和 $X^V$ 相同)和\keyterm{交叉注意力块}(其中 $X^K$ 和 $X^V$ 相同)。

值得注意的是,注意力算子,以及当没有屏蔽时的多头注意力层,对键和值的排列是不变的,并且与查询的排列\keyterm{等价},因为它会以类似的方式排列结果张量。

\section{Token 嵌入}\label{sec4.9}

在许多情况下,我们需要将离散 Token 转换为向量。这可以通过\keyterm{嵌入层}来完成,该层由一个直接将整数映射到向量的查找表组成。

嵌入层由两个\keyterm{元参数}定义:可能的 Token 值的数量 $N$,以及输出向量的维度 $D$,还一个 $N \times D$ 大小的可训练权重矩阵 $M$。

给定维度为 $D_1 \times \dots \times D_K$ 的整数张量 $X$ 和 ${0, \dots ,N -1}$ 中的值作为输入,嵌入层返回维度为 $D_1 \times \dots \times D_K \times D$ 的实值张量 $Y$,具体公式如下:
\begin{align*}
    \forall d_1, \dots, d_K,& \\
    Y[d&_1, \dots, d_K] = M[X[d_1, \dots, d_K]]
\end{align*}

\section{位置编码}\label{sec4.10}

虽然\keyterm{全连接层}的处理既特定于输入张量中特征的位置,也特定于输出张量中激活结果的位置,但\keyterm{卷积层}和\keyterm{多头注意力层}并不关心张量中的绝对位置。这是它们强大\keyterm{不变性}和\keyterm{归纳偏置}的关键,对于处理固定信号非常有益。

然而,在某些需要访问绝对位置才能进行恰当处理的情况下,这可能是个问题。例如,在图像合成中,场景的统计特性并不完全固定,或者在自然语言处理中,单词的相对位置强烈地调节了句子的含义。

解决这个问题的标准方法是在每个位置向特征表示中添加或连接一个\keyterm{位置编码},这是一个依赖于张量中位置的特征向量。这种位置编码可以像其他层参数一样被学习,或者通过分析来定义。

例如,在原始 \keyterm{Transformer} 模型中,对于一系列 $D$ 维向量,\cite{arxiv-1706.03762} 将序列索引的编码添加为一系列不同频率的正弦和余弦:
\begin{align*}
    \text{pos-enc}[t,d] &= \\
    &\begin{cases}
        \sin\Big(\frac{t}{T^{d/D}}\Big) &\text{如果}\; d \in 2\mathbb{N}\\
        \cos\Big(\frac{t}{T^{(d-1)/D}}\Big) &\text{否则}
    \end{cases}
\end{align*}
其中 $T= 10^4$
% !TeX root = ../main.tex
\chapter{架构}

\section{多层感知机}\label{sec5.1}

\section{卷积网络}\label{sec5.2}

\section{注意力模型}\label{sec:5.3}

%%%%%%%%%%%%%%%%%%%%%%%%%%%%%%%%%%%%%%%%%%%%%%%%%%%%%%%%%%%%%%%%%%%%%%%%%%%%%%%%%%%%%%%%%%%

\part{应用}

% !TeX root = ../main.tex
\chapter{预测}\label{ch6}

\section{图像去噪}\label{sec6.1}

\section{图像分类}\label{sec6.2}

\section{物体检测}\label{sec6.3}

\section{语义分割}\label{sec6.4}

\section{语音识别}\label{sec6.5}

\section{文本图像表示}\label{sec6.6}

\section{强化学习}\label{sec6.7}
% !TeX root = ../main.tex
\chapter{合成}\label{ch7}

与预测不同的第二类应用是合成。它包括将密度模型拟合到训练样本上,并提供从该模型中采样的方法。

\section{文本生成}\label{sec7.1}

\keyterm{文本合成}的标准方法是使用基于注意力的\keyterm{自动回归模型}。\cite{Radford2018} 提出了一个非常成功的模型,就是我们在 \ref{sec5.3} 节介绍的 \keyterm{GPT}。

该架构被用于构建超大规模的模型,例如 OpenAI 1750 亿参数的 GPT-3 \citep{arxiv-2005.14165}。它由 96 个自注意力模块构成,每个模块拥有 96 个头,能够处理 12,288 维的 Token,并且在 MLP 中拥有 49,512 隐藏维度。

当该模型在超大规模数据集上训练时,将得到\keyterm{大语言模型}(\keyterm{LLM}),其展现出极其强大的属性。它不仅整合了语言的词法和语法结构,还整合了多种多样的知识,例如预测``日本的首都是''、``如果将水加热到 100 摄氏度将变成''或``Jane 的小狗病了,所以她会''等句子后面的词。

这尤其能够解决\keyterm{小样本预测}问题,即只有少数训练示例可用,如图 \ref{fig7.1} 所示。更令人惊讶的是,当给出精心设计的\keyterm{提示词}时,它可以表现出回答问题、解决问题和思维链的能力,这些能力似乎非常接近高级推理 \citep{arxiv-2204.02311, arxiv-2303.12712}。

\begin{figure}
    \par{\fontsize{9pt}{11pt}
        \hrule
        ~ \\
        I: I love apples, O: positive, I: music is my passion, O: positive, I: my job is boring, O: negative, I: frozen pizzas are awesome, O: \textbf{positive,} \\
        \hrule
        ~ \\
        I: I love apples, O: positive, I: music is my passion, O: positive, I: my job is boring, O: negative, I: frozen pizzas taste like cardboard, O: \textbf{negative,} \\
        \hrule
        ~ \\
        I: water boils at 100 degrees, O: physics, I: the square root of two is irrational, O: mathematics, I: the set of prime numbers is infinite, O: mathematics, I: gravity is proportional to the mass, O: \textbf{physics,} \\
        \hrule
        ~ \\
        I: water boils at 100 degrees, O: physics, I: the square root of two is irrational, O: mathematics, I: the set of prime numbers is infinite, O: mathematics, I: squares are rectangles, O: \textbf{mathematics,} \\
        \hrule
        ~ \\
    }
    \caption[用 GPT 进行小样本预测]{小样本预测示例,采用来自 Hugging Face 的 1.2 亿参数的 GPT 模型。每个示例中,前面的句子是给定的\keyterm{提示词},模型生成的部分用粗体显示。}
    \label{fig7.1}
\end{figure}

由于其卓越的能力,这些模型有时被称为\keyterm{基础模型} \citep{arxiv-2108.07258}。

然而,即便它集成了大量的知识,这样的模型可能仍不足以满足实际应用,特别是在与人类用户交互时。在许多情况下,人们需要根据与助手进行的有效对话的统计数据做出响应。这与现有大规模训练集的统计数据不同,后者结合了小说、百科全书、论坛消息和博客文章。

这种差异可以通过对语言模型进行\keyterm{微调}来解决。目前主流策略是\keyterm{基于人类反馈的强化学习}(\keyterm{RLHF})\citep{arxiv-2203.02155},该策略包括通过要求用户编写回应或对生成的回应进行评分来创建小型标记训练集。前者可以直接用来对语言模型进行微调,而后者可以用来训练一个奖励网络,该网络能够预测评分,并将其作为目标,使用标准的\keyterm{强化学习}方法对语言模型进行微调。

由于语言模型架构大小的急剧增加,训练单一模型的成本可能高达数百万美元(见图 \ref{fig3.7}),因此,微调往往是在特定任务上实现高性能的唯一途径。

\section{图像生成}\label{sec7.2}

已经开发出多种深度方法来对高维密度进行建模和抽样。其中一个用于图像合成的强大方法依赖于反向\keyterm{扩散过程}。

其原理包括分析定义一个逐渐降解任何样本的过程,从而将数据中复杂且未知的密度转化为简单且众所周知的密度,例如正态分布,并训练一个深度架构来反转这一降解过程 \citep{arxiv-2006.11239}。

给定一个固定的 $T$,扩散过程定义了一个关于 $T + 1$ 张图像序列的概率分布,如下所示:从数据集中均匀采样 $x_0$,然后依次采样 $x_{t+1} \sim p(x_{t+1} \mid x_t), t = 0,\dots,T-1$,其中条件分布 $p$ 是分析定义的,并且逐渐擦除了 $x_0$ 中的结构。这种设置应该在很大程度上降解信号,以至于分布 $p(x_T )$ 有一个已知的分析形式,可以进行采样。

\begin{figure}
    \centering
    \includegraphics[width=0.9\textwidth]{fig/fig7.2.png}
    \caption[去噪扩散]{用去噪扩散进行图片合成 \citep{arxiv-2006.11239}。每个样本始于白噪音 $x_T$ (顶部),然后分别通过采样 $x_{t-1} \mid x_t \sim \mathcal{N} (x_t + f(x_t,t;w);\sigma_t)$ 逐步去噪。}
    \label{fig7.2}
\end{figure}

例如,\cite{arxiv-2006.11239} 将数据标准化,使其均值为 $0$,方差为 $1$,他们的扩散过程包括添加一点白噪声并将方差重新标准化为 $1$。这一过程指数级减少了 $x_0$ 的重要性,且 $x_t$ 的密度可以迅速用一个正态分布来近似。

去噪器 $f$ 是一个深度架构,其对 $f(x_{t-1},x_t,t;w) \simeq p(x_{t-1} \mid x_t)$ 进行建模并允许从中采样。得益于\keyterm{变分界限},可以证明,如果这个单步反向过程足够准确,那么从 $x_T \sim p(x_T)$ 采样并使用 $f$ 进行 $T$ 步去噪,将得到遵循 $p(x_0)$ 的 $x_0$。

训练 $f$ 可以通过生成大量的序列 $x_0^{(n)}, \dots , x_T^{(n)}$,在每个序列中选取一个 $t_n$,并最大化 
\[\sum_{n} \log f\Big(x_{t_n-1}^{(n)}, x_{t_n}^{(n)}, t_n; w\Big)\]
来实现。

在 \cite{arxiv-2006.11239} 给定的扩散过程中,去噪形式为:
\begin{equation}
    x_{t-1} \mid x_t \sim \mathcal{N} (x_t + f(x_t,t;w);\sigma_t) \label{eq7.1}
\end{equation}
其中 $\sigma_t$ 是以分析方式定义的。

在实践中,这种模型最初通过纯粹的运气在随机噪声中幻想出结构,然后通过强化到目前为止获得的图像最有可能的延续,逐渐构建出从噪声中浮现的更多元素。

这种方法可以扩展到文本条件合成,以生成与描述匹配的图像。例如,\cite{arxiv-2112.10741} 在方程 \ref{eq7.1} 的去噪分布的均值中加入了一个偏差,这个偏差的方向是增加产生的图像与条件文本描述之间的 CLIP 匹配得分(见 \ref{sec6.6} 章)。


%%%%%%%%%%%%%%%%%%%%%%%%%%%%%%%%%%%%%%%%%%%%%%%%%%%%%%%%%%%%%%%%%%%%%%%%%%%%%%%%%%%%%%%%%%%

\bibliography{main}


\printindex

%%%%%%%%%%%%%%%%%%%%%%%%%%%%%%%%%%%%%%%%%%%%%%%%%%%%%%%%%%%%%%%%%%%%%%%%%%%%%%%%%%%%%%%%%%%

\newpage

\vspace*{\stretch{1}}

\ifdefined\draft
\begin{center}
  {\color{red} (draft, do not circulate)}
\end{center}
\else
本书根据
\href{https://creativecommons.org/licenses/by-nc-sa/4.0/}{Creative
  Commons BY-NC-SA 4.0 国际许可}授权
\fi

\begin{center}
    翻译自 \href{https://fleuret.org/public/lbdl.pdf}{V1.1.1} - 2023.09.20
\end{center}

\vspace*{\stretch{1}}

%%%%%%%%%%%%%%%%%%%%%%%%%%%%%%%%%%%%%%%%%%%%%%%%%%%%%%%%%%%%%%%%%%%%%%%%%%%%%%%%%%%%%%%%%%%

\checknbdrafts

\end{document}
